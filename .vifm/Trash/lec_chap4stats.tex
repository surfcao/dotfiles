\documentclass[portrait]{seminar}
%\documentclass[portrait,article]{seminar}
%\portraitonly
%\articlemag{1}
%\def\printlandscape{\special{landscape}}  
\input{seminar.bug}
\input{seminar.bg2}
\RequirePackage{amssymb,amsfonts,bm} \RequirePackage{stmaryrd}
\RequirePackage{pstricks} \RequirePackage{graphicx,epsf}
\RequirePackage{semcolor,fancybox} \RequirePackage{slidesec}
\RequirePackage{fancyhdr}
%\RequirePackage{ae,latexsym}
%\RequirePackage{pslatex}
\setlength\slidewidth{8.5in} \setlength\slideheight{18cm}
\fancyhf{} % Clear all fields
\renewcommand{\headrulewidth}{0.25mm}
\renewcommand{\footrulewidth}{0.25mm}
\fancyhead[C]{\scriptsize \sffamily \textbf{GIST~4302:  Guofeng Cao}} %\fancyhead[L]{\vspace{0.1cm}
%\scalebox{0.07}{\includegraphics{geogLogoGray.eps}}}
\fancyhead[R]{\vspace{0.1cm}
\scalebox{0.1}{\includegraphics{TTU-seal-bw.pdf}}} \fancyfoot[L]{\tiny
\sffamily Lecture notes} \fancyfoot[C]{\tiny \sffamily \textbf{Point
Pattern Analysis I (Descriptive)}} \fancyfoot[R]{\tiny
\sffamily page \theslide/22}
% To avoid that the headers be too close of the top of the page
\renewcommand{\slidetopmargin}{1.5cm}
% To center horizontally the headers and footers (see seminar.bug)
\renewcommand{\headwidth}{\textwidth}
% To adjust the frame length to the header and footer ones
\autoslidemarginstrue

\newcommand{\matS}{\boldsymbol{\Sigma}}
\newcommand{\vecs}{\boldsymbol{\sigma}}
\newcommand{\vecmu}{\boldsymbol{\mu}}
\newcommand{\vecbeta}{\boldsymbol{\beta}}
\newcommand{\matL}{\boldsymbol{\Lambda}}
\newcommand{\vecl}{\boldsymbol{\lambda}}
\newcommand{\vectl}{\tilde{ \boldsymbol{ \lambda } } }
\newcommand{\mattL}{\tilde{ \boldsymbol{ \Lambda } } }
\newcommand{\vecm}{\boldsymbol{\mu}}
\newcommand{\slidetitle}[1]{\begin{center} \ptsize{10} \vspace*{-0.5cm} \doublebox{\textbf{#1}} \end{center}\ptsize{8}\vspace{0.4cm}}



\begin{document}
\pagestyle{fancy} \slideframe{} \sffamily
%
% New Page
%
\begin{slide*}
\begin{center}
\slidetitle{Descriptive vs Statitical Point Pattern Analysis}
%\ptsize{10} \vspace*{-0.5cm} {\S \textbf{Descriptive vs
%Statistical Point Pattern Analysis}}
\end{center}
\ptsize{8} \vspace{0.4cm}

\underline{\textbf{Descriptive analysis:}}
\begin{itemize}
\item set of quantitative (and graphical) tools for characterizing
spatial point patterns
\item different tools are appropriate for investigating first- or
second-order effects (e.g., kernel density estimation versus sample
G function)
\item  can shed light onto whether points are clustered or evenly
distributed in space
\end{itemize}

\vspace{0.2cm} \underline{\textbf{Limitation:}}
\begin{itemize}
\item no assessment of \emph{\underline{how}} clustered or
\emph{\underline{how}} evenly-spaced is an
observed point pattern
\item no yardstick against which to compare observed values (or
graph) of results
\end{itemize}

\vspace{0.3cm} \underline{\textbf{Statistical analysis:}}
\begin{itemize}
\item assessment of whether an observed point pattern
can be regarded as one (out of many) realizations from a particular
spatial process
\item measures of confidence with which the above
assessment can be made (how likely is that the observed pattern is a
realization of a particular spatial process)
\end{itemize}


\end{slide*}

%
% New Page
%
\begin{slide*}
\begin{center}
\ptsize{10} \vspace*{-0.5cm} \fbox{\textbf{Some General
Terminology}}
\end{center}
\ptsize{8} \vspace{0.4cm}

\underline{\textbf{Null hypothesis:}}
\begin{itemize}
\item spatial process that is hypothesized to be the generating
mechanism of spatial patterns
\item in this class, we'll focus on the null hypothesis of complete
spatial randomness (CSR)
\end{itemize}

\vspace{0.3cm} \underline{\textbf{Samples:}}
\begin{itemize}
\item realizations from the above process
\item in this lecture, the set of alternative point patterns
that could result from the null process of CSR
\end{itemize}

\vspace{0.3cm} \underline{\textbf{Sampling distribution of a
statistic:}}
\begin{itemize}
\item \emph{sample statistic} = a summary measure, e.g., mean or
entire CDF, characterizing (and thus computed from) a sample
\item \emph{sampling distribution of a statistic} =
distribution, \\ e.g., histogram, of such a summary measure computed
from many alternative samples generated from the null process
\end{itemize}


\end{slide*}

%
% New Page
%
\begin{slide*}
\begin{center}
\ptsize{10} \vspace*{-0.5cm} \fbox{\textbf{A Particular Example I}}
\end{center}
\ptsize{8} \vspace{0.3cm}

\underline{\textbf{Null hypothesis:}}
\begin{itemize}
\item complete spatial randomness (CSR) as a mechanism for
generating point patterns
\end{itemize}

\vspace{0.3cm} \underline{\textbf{Statistic:}}
\begin{itemize}
\item mean event-to-nearest-event (E2NE) distance;
\\ here the variable is the minimum distance between events (E2NE),
and the selected summary statistic is the mean of those distances:
\[
\bar{d}_{min} = \frac{1}{n} \sum_{i=1}^n d_{min}({\bf s}_i)
\]
\begin{center}
{\small $d_{min}({\bf s}_i)$ = distance between $i$-th event and its
nearest neighbor event}
\end{center}
\end{itemize}

\vspace{0.3cm} \underline{\textbf{Sampling distribution of mean
E2NE:}}
\begin{enumerate}
\item generate (simulate) one realization of a point pattern under CSR
\item compute $\bar{d}_{min}$ from that realization
\item repeat steps (1) and (2) many, say 1000, times
\item histogram of 1000 $\bar{d}_{min}$ values is the sampling
distribution of mean E2NE distance \emph{\underline{under the null
hypothesis of CSR}}
\end{enumerate}

\vspace{0.2cm}
\begin{center}
\emph{sampling distribution under CSR can often be analytically
derived, without resorting to simulation}
\end{center}

\end{slide*}

%
% New Page
%
\begin{slide*}
\begin{center}
\ptsize{10} \vspace*{-0.5cm} \fbox{\textbf{A Particular Example II}}
\end{center}
\ptsize{8} \vspace{0.3cm}

\underline{\textbf{Two realizations under CSR}} of point patterns
with $n=50$ events:

\vspace{0.1cm}
\begin{center}
\begin{figure}
\scalebox{0.45}{\includegraphics{FsimCSRn50A.eps}} \hspace{0.5cm}
\scalebox{0.45}{\includegraphics{FsimCSRn50B.eps}}
\end{figure}
\end{center}
\vspace{-0.4cm} {\small \hspace{1.6cm} $\bar{d}_{min} = 7.13$
\hspace{2.2cm} $\bar{d}_{min} = 8.13$}

\vspace{0.3cm}\underline{\textbf{Sampling distribution}} or
histogram of $\bar{d}_{min}$ values from $500$ simulated (under CSR)
point patterns with $n=50$ events

\vspace{0.0cm}
\begin{center}
\begin{figure}
\scalebox{0.55}{\includegraphics{FsimCSRn50bardminSD.eps}}
\end{figure}
\end{center}

\end{slide*}

%
% New Page
%
\begin{slide*}
\begin{center}
\ptsize{10} \vspace*{-0.5cm} \fbox{\textbf{A Particular Example
III}}
\end{center}
\ptsize{8} \vspace{0.3cm}

\underline{\textbf{Two realizations under CSR}} of point patterns
with $n=100$ events:

\vspace{0.1cm}
\begin{center}
\begin{figure}
\scalebox{0.45}{\includegraphics{FsimCSRn100A.eps}} \hspace{0.5cm}
\scalebox{0.45}{\includegraphics{FsimCSRn100B.eps}}
\end{figure}
\end{center}
\vspace{-0.4cm} {\small \hspace{1.6cm} $\bar{d}_{min} = 4.88$
\hspace{2.2cm} $\bar{d}_{min} = 5.67$}

\vspace{0.3cm}\underline{\textbf{Sampling distribution}} or
histogram of $\bar{d}_{min}$ values from $500$ simulated (under CSR)
point patterns with $n=100$ events

\vspace{0.0cm}
\begin{center}
\begin{figure}
\scalebox{0.55}{\includegraphics{FsimCSRn100bardminSD.eps}}
\end{figure}
\end{center}

\end{slide*}

%
% New Page
%
\begin{slide*}
\begin{center}
\ptsize{10} \vspace*{-0.7cm} \fbox{\textbf{Looking at Observed Point
Patterns I}}
\end{center}
\ptsize{8} \vspace{0.1cm}

\underline{\textbf{Two observed point patterns}} with $n=100$
events:

\vspace{0.0cm}
\begin{center}
\begin{figure}
\scalebox{0.45}{\includegraphics{points_randstrat.eps}}
\hspace{0.5cm} \scalebox{0.45}{\includegraphics{points_clust.eps}}
\end{figure}
\end{center}
\vspace{-0.4cm} {\small \hspace{1.6cm} $\bar{d}_{min} = 7.86$
\hspace{2.2cm} $\bar{d}_{min} = 1.17$}

\vspace{0.1cm}\underline{\textbf{Question:}} Could these two
patterns be realizations under CSR?

\vspace{-0.2cm}
\begin{center}
\begin{figure}
\scalebox{0.55}{\includegraphics{FsimCSRn100bardminSD.eps}}
\end{figure}
\end{center}

\vspace{-0.3cm}
\begin{center}
\emph{Obviously \underline{no}, and this can be said with great
confidence; \\ {\small pattern on left has much larger mean E2NE
distance than expected under CSR, and vise versa for pattern on
right}}
\end{center}

\end{slide*}

%
% New Page
%
\begin{slide*}
\begin{center}
\ptsize{10} \vspace*{-0.6cm} \fbox{\textbf{Looking at Observed Point
Patterns II}}
\end{center}
\ptsize{8} \vspace{0.2cm}

\underline{\textbf{Observed point pattern}} with $n=100$ events:

\begin{center}
\begin{figure}
\scalebox{0.45}{\includegraphics{points_rand.eps}}
\end{figure}
\end{center}
\vspace{-0.4cm} {\small \hspace{3.6cm} $\bar{d}_{min} = 5.18$}

\vspace{0.2cm}\underline{\textbf{Question:}} Is this pattern more
clustered than a CSR-generated one?

\begin{center}
\begin{figure}
\scalebox{0.50}{\includegraphics{FsimCSRn100bardminSDline.eps}}
\end{figure}
\end{center}

\vspace{-0.2cm}
\begin{center}
\emph{Most probably \underline{no}, since observed
$\bar{d}_{min}=5.18$ (black bar) \\ lies at the center of the
sampling distribution under CSR}
\end{center}

\end{slide*}


%
% New Page
%
\begin{slide*}
\begin{center}
\ptsize{10} \vspace*{-0.6cm} \fbox{\textbf{Looking at Observed Point
Patterns III}}
\end{center}
\ptsize{8} \vspace{0.2cm}

\underline{\textbf{Observed point pattern}} with $n=100$ events, and
sampling distribution of $\bar{d}_{min}$ under CSR:

\begin{center}
\begin{figure}
\scalebox{0.45}{\includegraphics{FxyRandClus.eps}} \hspace{0.5cm}
\scalebox{0.45}{\includegraphics{FsimCSRn100bardminSDline2.eps}}
\end{figure}
\end{center}
\vspace{-0.5cm} {\small \hspace{1.45cm} $\bar{d}_{min} = 4.65$}

\vspace{0.2cm}\underline{\textbf{Question:}} Is this pattern more
clustered than a CSR-generated one?

\vspace{0.2cm}\underline{\textbf{Equivalent question:}} Since small
$\bar{d}_{min}$ values indicate clustering, is observed
$\bar{d}_{min}$ on \underline{left} side of sampling distribution
under CSR?

\vspace{0.2cm}\underline{\textbf{Answer:}} The area under the curve
of the sampling distribution to the \underline{left} of observed
$\bar{d}_{min}=4.65$ is an indication of how \underline{unlikely} is
the observed pattern to be generated by CSR: \emph{the
\underline{smaller} that area, the \underline{more unlike} is the
pattern to be a realization under CSR}

{\small \underline{NOTE:} if we were asking whether the observed
pattern was more \underline{even} (less clustered) than a
CSR-generated one, we would be looking at the area under the curve
to the \underline{right} of observed $\bar{d}_{min}=4.65$, since we
would be interested in larger (than CSR-related) such distance
values}

\end{slide*}

%
% New Page
%
\begin{slide*}
\begin{center}
\ptsize{10} \vspace*{-0.6cm} \fbox{\textbf{P-Value of An Observed
Sample Statistic}}
\end{center}
\ptsize{8} \vspace{0.1cm}

\begin{center}
\begin{figure}
\scalebox{0.50}{\includegraphics{FsimCSRn100bardminSDline2.eps}}
\hspace{0.3cm}
\scalebox{0.50}{\includegraphics{FsimCSRn100bardminSDcdf2.eps}}
\end{figure}
\end{center}

\vspace{-0.2cm}\underline{\textbf{P-value}} = Area under the curve
of the sampling distribution to the direction of the alternative
hypothesis from the observed statistic \\
$\longrightarrow$ probability of observing a $\bar{d}_{min}$ value
$\leq$ $4.65$ in this case

{\small \underline{NOTE:} direction dependence in defining the
$P$-value comes into play for \underline{one-sided} tests; when we
are just interested in whether the null hypothesis holds or not, no
matter the direction of the alternative hypothesis
(\underline{two-sided test}), the final $P$-value is defined as
\underline{twice} the above $P$-value}

\vspace{0.2cm}\underline{\textbf{Interpretation:}} The $P$-value is
an indication of how \underline{unlikely} is the observed pattern to
be generated by the null hypothesis: \emph{the \underline{smaller}
the $P$-value, the \underline{more unlike} is the pattern to be a
realization under the null hypothesis, here CSR}

\vspace{0.2cm}\underline{\textbf{Note:}} Any $P$-value is associated
with a null hypothesis, since a $P$-value is computed from a
sampling distribution which in turn is generated under a null
hypothesis

\end{slide*}

%
% New Page
%
\begin{slide*}
\begin{center}
\ptsize{10} \vspace*{-0.5cm} \fbox{\textbf{Sampling Distribution of
G Function Under CSR}}
\end{center}
\ptsize{8}


\vspace{0.5cm} \underline{\textbf{Sampling distribution:}} of
$\hat{G}(d)$ under CSR computed from $500$ simulated point patterns
within a square region of area $|A| = 100\times 100$:

\begin{center}
\begin{figure}
\hspace{-0.3cm}
\scalebox{0.50}{\includegraphics{FghatCSRn100SD.eps}} \hspace{0.5cm}
\scalebox{0.50}{\includegraphics{FghatCSRn50SD.eps}}
\end{figure}
\end{center}

\vspace{0.3cm} \underline{\textbf{Interpretation:}} Plots provide
envelope of simulated minimum and maximum $G(d)$ values for
assessing whether an observed point pattern (not available here) can
be regarded a realization from a CSR null process; \\ this is done
by: (i) comparing the observed $\hat{G}(d)$ value (not available
here) with the expected (mean) curve, and (ii) assessing its
relative position within the envelope

\vspace{0.5cm}
\begin{center}
\emph{The larger $n$ is (more events in the domain), the tighter the
envelop}
\end{center}


\end{slide*}

%
% New Page
%
\begin{slide*}
\begin{center}
\ptsize{10} \vspace*{-0.5cm} \fbox{\textbf{Assessing Observed Ghat
Plots I}}
\end{center}
\ptsize{8}


\vspace{0.2cm} \underline{\textbf{Two observed point patterns}} with
$n=100$ events:

\vspace{0.0cm}
\begin{center}
\begin{figure}
\scalebox{0.45}{\includegraphics{points_randstrat.eps}}
\hspace{0.5cm} \scalebox{0.45}{\includegraphics{points_clust.eps}}
\end{figure}
\end{center}

\vspace{0.2cm}\underline{\textbf{Question:}} Could these two
patterns be realizations under CSR?

\begin{center}
\begin{figure}
\hspace{-0.3cm}
\scalebox{0.50}{\includegraphics{FghatCSRn100SDstrat.eps}}
\hspace{0.5cm}
\scalebox{0.50}{\includegraphics{FghatCSRn100SDclus.eps}}
\end{figure}
\end{center}

\vspace{0.0cm}
\begin{center}
\emph{Most probably \underline{no}, since the observed $\hat{G}(d)$
curve lies outside the simulation envelope}
\end{center}

\end{slide*}

%
% New Page
%
\begin{slide*}
\begin{center}
\ptsize{10} \vspace*{-0.5cm} \fbox{\textbf{Assessing Observed Ghat
Plots II}}
\end{center}
\ptsize{8}


\vspace{0.2cm} \underline{\textbf{Observed point patterns}} with
$n=100$ events:

\vspace{0.0cm}
\begin{center}
\begin{figure}
\scalebox{0.45}{\includegraphics{points_rand.eps}}
\end{figure}
\end{center}

\vspace{0.1cm}\underline{\textbf{Question:}} Could this pattern be a
realization under CSR?

\begin{center}
\begin{figure}
\scalebox{0.50}{\includegraphics{FghatCSRn100SDrand.eps}}
\end{figure}
\end{center}

\vspace{-0.0cm}
\begin{center}
\emph{Most probably \underline{yes}, since the observed $\hat{G}(d)$
curve lies very close to the mean simulated plot and is well within
the simulation envelope}
\end{center}

\end{slide*}

%
% New Page
%
\begin{slide*}
\begin{center}
\ptsize{10} \vspace*{-0.5cm} \fbox{\textbf{Analytically-Derived
Sampling Distributions}}
\end{center}
\ptsize{8}


\vspace{0.3cm} \underline{\textbf{Analytical derivations:}}
\begin{itemize}
\item for simple domains, e.g., rectangles, there are mathematical
formulae that provide the expected values of sample statistics under
CSR
\item in other words, people have already calculated what is the
mean of a very large number of simulated $\bar{d}_{min}$ or
$\hat{G}(d)$ values under CSR, without ever touching a computer
\item these formulae have been derived before the advent of powerful
computers, and have been used for a long time in point pattern
analysis
\item since, no simulation runs are involved, such
analytically-derived formulae can be easily used without the need to
resort to computer-intensive simulation procedures
\end{itemize}

\vspace{0.3cm} \underline{\textbf{Limitations:}}
\begin{itemize}
\item analytically-derived formulae need to account for the fact
that events near the boundary of the study region do not have the
same number of neighbors as events in the middle of that region
\item such \underline{edge effects} can be taken care of when
the study region has simple geometry, e.g., for rectangles
\end{itemize}

\vspace{0.2cm}
\begin{center}
\emph{In general, if you have access to computer software that can
perform simulation, do not use analytically-derived formulae...}
\end{center}


\end{slide*}


%
% New Page
%
\begin{slide*}
\begin{center}
\ptsize{10} \vspace*{-0.5cm} \fbox{\textbf{CSR-Expected Mean Nearest
Neighbor Distance}}
\end{center}
\ptsize{8}

\vspace{0.3cm}
\begin{itemize}
\item \textbf{\underline{Definition}:} average of all
$d_{min}({\bf s}_i)$ values:
\[
\bar{d}_{min} = \frac{1}{n} \sum_{\alpha=1}^n d_{min}({\bf s}_i)
\]
\item single number does not suffice to describe point pattern
\end{itemize}


\vspace{0.3cm} \underline{\textbf{Checking for CSR:}}
\begin{enumerate}
\item compute expected value of mean nearest-neighbor
distance, under CSR:
\[
E\{ \bar{d}_{min} \} = \frac{1} { 2 \sqrt{\lambda} }
\]
{\small $\lambda$ = overall intensity of point pattern = \\
(\# of points within study region) / (area of region)}
\item form ratio $R$: $\; \;$
\fbox{ $R = \frac{ \bar{d}_{min} }{ 1 / ( 2 \sqrt{\lambda} ) } = 2
\bar{d}_{min} \sqrt{\lambda} $} \vspace{0.3cm}
\item \underline{interpretation:}
$R < 1$ $\Rightarrow$ observed nearest neighbor distances shorter
than expected $\Rightarrow$ tendency
towards clustering \\
$R > 1$ $\Rightarrow$ tendency towards evenly spaced events
\end{enumerate}

\vspace{0.3cm}
\begin{center}
\emph{Result depends heavily upon study area definition ({\small
used to compute $\lambda$})}
\end{center}

\end{slide*}

%
% New Page
%
\begin{slide*}
\begin{center}
\ptsize{10} \vspace*{-0.5cm} \fbox{\textbf{CSR-Expected G and F
Functions}}
\end{center}
\ptsize{8}

\vspace{0.3cm} \underline{\textbf{G function definition:}}
\begin{itemize}
\item proportion of event-to-nearest-neighbor distances
$d_{min}({\bf s}_\alpha)$ no greater than given distance cutoff $d$
\item cumulative distribution function (CDF)
of all $n$ event-to-nearest-event distances:
\[
\hat{G}(d) = \frac{ \# [ d_{min}({\bf s}_i) \leq d] }{n}
\]
\end{itemize}

\vspace{0.3cm} \underline{\textbf{F function definition:}}
\begin{itemize}
\item proportion of point-to-nearest-neighbor distances
$\tilde{d}_{min}(\tilde{{\bf s}}_p)$ no greater than given distance
cutoff $d$
\item cumulative distribution function (CDF)
of all $m$ point-to-nearest-event distances:
\[
\hat{F}(d) = \frac{ \# [ \tilde{d}_{min}(\tilde{{\bf s}}_p) \leq d]
}{m}
\]
\end{itemize}

\vspace{0.3cm} \underline{\textbf{Expected G and F function under
CSR:}}
\[E\{G(d)\} =
E\{F(d)\} = 1 - e^{-\lambda \pi d^2 }\]

\vspace{0.1cm} \underline{\textbf{Checking for CSR:}}

compare empirical functions $\hat{G}(d)$ and $\hat{F}(d)$ with their
theoretical counterparts $E\{ G(d) \}$ and $E\{ F(d) \}$ under CSR


\end{slide*}

%
% New Page
%
\begin{slide*}
\begin{center}
\ptsize{10} \vspace*{-0.5cm} \fbox{\textbf{Examples of G Functions}}
\end{center}
\ptsize{8}

\vspace{0.1cm}
\begin{center}
\begin{figure}
\scalebox{0.45}{\includegraphics{points_rand.eps}} \hspace{0.5cm}
\scalebox{0.45}{\includegraphics{points_clust.eps}}
\end{figure}
\end{center}

\vspace{0.1cm}
\begin{center}
\begin{figure}
\hspace{-0.3cm} \scalebox{0.45}{\includegraphics{FghatRandExp.eps}}
\hspace{0.5cm}
\scalebox{0.45}{\includegraphics{FghatClusterExp.eps}}
\end{figure}
\end{center}

\begin{center}
{\small solid lines indicate expected value of $G(d)$ under CSR:
$E\{G(d)\} = 1 - e^{-0.01 \pi d^2 }$ }
\end{center}

\vspace{0.1cm}
\begin{itemize}
\item for {\em clustered events}, $\hat{G}(d)$ rises sharply
at short distances, and levels off at large $d$-values
\end{itemize}

\end{slide*}


%
% New Page
%
\begin{slide*}
\begin{center}
\ptsize{10} \vspace*{-0.5cm} \fbox{\textbf{Examples of F Functions}}
\end{center}
\ptsize{8}

\vspace{0.1cm}
\begin{center}
\begin{figure}
\scalebox{0.45}{\includegraphics{points_rand.eps}} \hspace{0.5cm}
\scalebox{0.45}{\includegraphics{points_clust.eps}}
\end{figure}
\end{center}

\vspace{0.1cm}
\begin{center}
\begin{figure}
\hspace{-0.3cm} \scalebox{0.45}{\includegraphics{FfhatRandExp.eps}}
\hspace{0.5cm}
\scalebox{0.45}{\includegraphics{FfhatClusterExp.eps}}
\end{figure}
\end{center}

\begin{center}
{\small solid lines indicate expected value of $F(d)$ under CSR:
$E\{F(d)\} = 1 - e^{-0.01 \pi d^2 }$ }
\end{center}

\vspace{0.1cm}
\begin{itemize}
\item for {\em clustered events}, $\hat{F}(d)$ rises
slowly at short distances, and more rapidly at longer distances
\end{itemize}

\end{slide*}


%
% New Page
%
\begin{slide*}
\begin{center}
\ptsize{10} \vspace*{-0.5cm} \fbox{\textbf{Example with Evenly
Spaced Points}}
\end{center}
\ptsize{8}

\vspace{0.1cm}
\begin{center}
\begin{figure}
\scalebox{0.45}{\includegraphics{points_randstrat.eps}}
\end{figure}
\end{center}

\vspace{0.1cm}
\begin{center}
\begin{figure}
\hspace{-0.3cm}
\scalebox{0.45}{\includegraphics{FghatRandStratExp.eps}}
\hspace{0.5cm}
\scalebox{0.45}{\includegraphics{FfhatRandStratExp.eps}}
\end{figure}
\end{center}

\begin{center}
{\small solid lines indicate expected value of $G(d)$ and $F(d)$
under CSR: $1 - e^{-0.01 \pi d^2 }$ }
\end{center}

\vspace{0.3cm}
\begin{itemize}
\item for {\em evenly-spaced events}, $\hat{G}(d)$ rises slowly
at short distances, and then increases rapidly
\end{itemize}


\end{slide*}

%
% New Page
%
\begin{slide*}
\begin{center}
\ptsize{10} \vspace*{-0.5cm} \fbox{\textbf{The K Function}}
\end{center}
\ptsize{8}
\begin{center}
Looking \underline{\emph{beyond}} nearest neighbors
\end{center}

\vspace{0.2cm} \underline{\textbf{Concept:}}
\begin{enumerate}
\item construct set of concentric circles
(of increasing radius $d$) around each event
\item count number of events in each distance ``band''
\item cumulative number of events up to radius $d$
around {\it \underline{all}} events becomes the sample $K$ function
$\hat{K}(d)$
\end{enumerate}

\vspace{0.0cm}
\begin{center}
\begin{figure}
\scalebox{0.30}{\includegraphics{Fkfunction.eps}}
\end{figure}
\end{center}

\vspace{0.1cm} \underline{\textbf{Formal definition:}}
\begin{eqnarray*}
K(d) & = & \frac { E\{ \mbox{ {\small \# of events within distance
$d$ of any arbitrary event} } \} }
{ E\{ \mbox{ {\small \# of events within study area} } \} } \\
& \simeq & \frac{1}{\lambda} \frac{1}{n} \#\{d_{ij} \leq d,
i=1,\ldots,n, j=1,\ldots,n \} = \hat{K}(d)
\end{eqnarray*}

\end{slide*}

%
% New Page
%
\begin{slide*}
\begin{center}
\ptsize{10} \vspace*{-0.5cm} \fbox{\textbf{CSR-Expected K Function}}
\end{center}
\ptsize{8}

\vspace{0.5cm} \underline{\textbf{Under CSR:}}
\begin{itemize}
\item $E\{ K(d) \} = \frac{ \lambda \pi d^2 }{\lambda} = \pi d^2$
\item this can become a very large number (due to $d^2$),
and consequently small differences between $\hat{K}(d)$ and $E\{
K(d) \}$ cannot be easily resolved
\item use $L$ function instead:
\[
\hat{L}(d) = \sqrt{ \frac{\hat{K}(d)}{\pi} } - d
\]
with $E\{ L(d) \} =0$
\end{itemize}

\vspace{0.5cm} \underline{\textbf{Interpreting the L function:}}
\begin{itemize}
\item for $\hat{L}(d) > 0$ $\Rightarrow$ more events
separated by distance $d$ than expected under CSR $\Rightarrow$ {\em
clustered events}
\item watch out for edge effects $\ldots$
\end{itemize}

\end{slide*}


%
% New Page
%
\begin{slide*}
\begin{center}
\ptsize{10} \vspace*{-0.5cm} \fbox{\textbf{Examples of L Functions}}
\end{center}
\ptsize{8}

\vspace{0.1cm}
\begin{center}
\begin{figure}
\scalebox{0.45}{\includegraphics{points_rand.eps}} \hspace{0.5cm}
\scalebox{0.45}{\includegraphics{points_clust.eps}}
\end{figure}
\end{center}

\vspace{-0.9cm}
\begin{center}
\hspace{-0.6cm}
\begin{figure}
\scalebox{0.45}{\includegraphics{FlhatRand.eps}} \hspace{0.3cm}
\scalebox{0.45}{\includegraphics{FlhatClus.eps}}
\end{figure}
\end{center}

\vspace{0.0cm} \underline{\textbf{Expected appearance:}}
\begin{itemize}
\item for $\hat{L}(d) > 0$ $\Rightarrow$ more events
separated by distance $d$ than expected under CSR $\Rightarrow$ {\em
clustered events}
\item watch out for edge effects $\ldots$
\end{itemize}


\end{slide*}

%
% New Page
%
\begin{slide*}
\begin{center}
\ptsize{10} \vspace*{-0.5cm} \fbox{\textbf{Recap}}
\end{center}
\ptsize{8}

\vspace{0.5cm}\underline{\textbf{Statistical analysis of spatial
point patterns:}}
\begin{itemize}
\item allows to quantify departure of results obtained via
exploratory tools, e.g., $\bar{d}_{min}$ or $\hat{G}(d)$, from
expected such results derived under specific null hypotheses, here
CSR hypothesis
\item can be used to assess to what extent observed point patterns
can be regarded as realizations from a particular spatial process
(here CSR)
\end{itemize}

\vspace{0.5cm} \underline{\textbf{Sampling distribution of a test
statistic:}}
\begin{itemize}
\item lies at the heart of any statistical hypothesis testing
procedure, and is tied to a particular null hypothesis
\item simulation and analytical derivations are two alternative ways
of computing such sampling distributions (the latter being
increasingly replaced by the former)
\end{itemize}

\vspace{0.5cm}
\begin{center}
Watch out for \emph{\underline{edge effects}...}
\end{center}



\end{slide*}

\end{document}
